\documentclass[tikz,border=10pt,dvipsnames]{standalone}
\usetikzlibrary{
  arrows.meta,                        % for arrow tips
  backgrounds,                        % for background layer
  bending,
  ext.paths.ortho,                    % for ortho paths
  positioning,
  ext.positioning-plus,               % for 
  ext.node-families.shapes.geometric, % loads ext.node-families and
% shapes.geometric,                   % for ellipse
  calc,                               % for ($$)
  backgrounds %https://tex.stackexchange.com/questions/230224/how-to-change-the-background-color-in-tikz
}
% node styles
\tikzstyle{flowstep} = [rectangle, draw, text centered, rounded corners, minimum height=2em]
\tikzstyle{question} = [ellipse, draw, text centered, rounded corners]
\tikzstyle{semistep}=[flowstep,fill=green!20]
\tikzstyle{fullstep}=[flowstep,fill=blue!20]
\tikzstyle{concept}=[flowstep,fill=orange!20]
\tikzstyle{simstep}=[flowstep,fill=yellow!20]
\tikzstyle{veristep}=[flowstep,fill=BurntOrange!60]
\tikzstyle{designstep}=[flowstep,fill=pink!50]
\tikzstyle{expensivestep}=[flowstep,fill=red!30]
% edge styles
\tikzstyle{arr}=[->,rounded corners,thick]
\tikzstyle{dottarr}=[->,rounded corners,dotted]
\tikzstyle{happypath}=[->,rounded corners,thick,ForestGreen!75!white]
% no arrow
\tikzstyle{unhappypathna}=[rounded corners,thick,BrickRed!75!white]
\tikzstyle{unhappypath}=[->,unhappypathna]
\tikzstyle{veryunhappypath}=[rounded corners,thick,Red,->]
\tikzset{
  fill fraction/.style n args={3}{path picture={
 \fill[#1] (path picture bounding box.south west) rectangle
 ($(path picture bounding box.north west)!#3!(path picture bounding box.north
 east)$);
 \fill[#2] (path picture bounding box.south east) rectangle
 ($(path picture bounding box.north west)!#3!(path picture bounding box.north
 east)$);}}
}

\begin{document}
%source: andreas burg https://ee222-winter19-01.courses.soe.ucsc.edu/system/files/attachments/EE222W19_Lect1_Part2-Introduction.pdf

% TODO add an image to each level, unveil in 2nd step all at once
\begin{tikzpicture}
  % first flow nodes
\node[concept](conceptidea) at (3,0) {Behavioural specification};
\node[flowstep,fill fraction={green!20}{blue!20}{0.5}](layout)[below=of conceptidea]{Layout Creation}; % can either be full custom or semi custom, will involve netlist,floor plan place and route
% semi custom
\node[semistep](semicustom) [below left=of layout,fill=green!20] {Semi-Custom-Design};
\node[semistep](semiopt) [below =of semicustom,fill=green!20] {Optimization};
\node[semistep](semimap) [below =of semiopt,fill=green!20] {Technology Mapping};
\node[semistep](semipr) [below =of semimap,fill=green!20] {Place + Route};
%full custom
\node[fullstep](fullcustom) [below right=of layout,fill=blue!20] {Full-Custom-Design};
\node[fullstep](fullopt) [below =of fullcustom,fill=blue!20] {Topology Selection};
\node[fullstep](fullmap) [below =of fullopt,fill=blue!20] {Sizing};
\node[fullstep](fullpr) [below =of fullmap,fill=blue!20] {Manual Layouting};
\node (centerfin) at ($(semipr)!0.5!(fullpr)$){};
\node[simstep](layoutverif) [below =of centerfin] {Layout Verification (DRC+LVS)};
\node[simstep](postsim) [below =of layoutverif] {Post Layout Simulation};

\node[question] (meetsspec) [below=of postsim]{Meets specificaton?};
\node[flowstep] (iterate) [below left=of meetsspec] {Iterate the process};
\node[expensivestep] (tapeout) [below right=of meetsspec] {Tapeout};
% draw arrows
\draw[arr] (conceptidea) -- (layout);
\draw[arr,green!40] (layout.west) -| (semicustom.north);
\draw[arr,green!40] (semicustom) -- (semiopt);
\draw[arr,green!40] (semiopt) -- (semimap) ;
\draw[arr,green!40] (semimap) -- (semipr) ;
\draw[arr] (semipr) |- (layoutverif.west);

\draw[arr,blue!40] (layout.east) -| (fullcustom.north);
\draw[arr,blue!40] (fullcustom) -- (fullopt);
\draw[arr,blue!40] (fullopt) -- (fullmap) ;
\draw[arr,blue!40] (fullmap) -- (fullpr) ;
\draw[arr] (fullpr) |- (layoutverif.east);
%

\draw[dottarr] (layoutverif.north) |- (fullpr);
\draw[dottarr] (layoutverif.north) |- (fullmap);
\draw[dottarr] (layoutverif.north) |- (fullopt);
\draw[dottarr] (layoutverif.north) |- (semipr);
\draw[dottarr] (layoutverif.north) |- (semimap);
\draw[dottarr] (layoutverif.north) |- (semiopt);

\draw[arr] (layoutverif) -- (postsim);
\draw[arr] (postsim) -- (meetsspec);
\draw[arr,red] (meetsspec) -| (iterate.north) node [midway, above]{No};
\draw[arr,green] (meetsspec) -| (tapeout.north) node [midway, above]{Yes};
\node (ctrlPR) [left=of semipr];
\draw[dottarr] (iterate.west) -| (ctrlPR) |- (layout.west);
\draw[dottarr] (iterate.west) -| (ctrlPR) |- (conceptidea.west);

% draw a dot around the shared sections
% this is the same, and what we can optimize
\node (bothsynth) at ($(semiopt.north west)!0.5!(fullopt.north east) + (0.0,0.25)$) {Topology Synthesis};
\draw[dotted] ($(semiopt.north west)+(-1.1,0.1)$) -|($(fullopt.north east)+(0.1,0.1)$)  |- ($(fullmap.south east)+(0.1,-0.25)$) -| ($(semimap.south west)+(-0.50,-0.25)$) -- ($(semiopt.north west)+(-1.1,0.1)$);



%% in semicustom: synthesizeable behavioural sim, synthesis (opt + map), place + route
% in full custom: decide on topology,size, manual place + route
% => same steps, just 
% *very* unhappy paths manual/finegrained vs automated
% we already have good P+R and they are getting better
% how do we think about topologies as a unit instead of hierarchical macros?
\end{tikzpicture}
\end{document}
