% !TeX root = ../../presentation.tex
\documentclass{standalone}
\usepackage[utf8]{inputenc}
\usepackage{graphicx}
% --- Fonts ---
\usepackage[french,german,english]{babel}
\usepackage{lmodern} % use this to fix blurry typewriter text font
%\usepackage[scaled=1.24]{nimbusmononarrow}% typewriter font
\usepackage{nimbusmononarrow}% typewriter font
\usepackage{notomath}
\usepackage[T1]{fontenc}
% tikz definitions
\usepackage[dvipsnames]{xcolor}
\usepackage{tikz}
\usetikzlibrary{
  arrows.meta,                        % for arrow tips
  backgrounds,                        % for background layer
  bending,
  ext.paths.ortho,                    % for ortho paths
  positioning,
  ext.positioning-plus,               % for 
  ext.node-families.shapes.geometric, % loads ext.node-families and
% shapes.geometric,                   % for ellipse
  calc,                               % for ($$)
  backgrounds, %https://tex.stackexchange.com/questions/230224/how-to-change-the-background-color-in-tikz
  overlay-beamer-styles,
}
\newcommand{\visible}[1]{#1}
\newcommand{\only}[1]{#1}

% thanks https://tex.stackexchange.com/a/62032
\usepackage{xparse}

\ExplSyntaxOn

\NewDocumentCommand{\getenv}{om}
 {
  \sys_get_shell:nnN { kpsewhich ~ --var-value ~ #2 } { } \l_tmpa_tl
  \tl_trim_spaces:N \l_tmpa_tl
  \IfNoValueTF { #1 }
   {
    \tl_use:N \l_tmpa_tl
   }
   {
    \tl_set_eq:NN #1 \l_tmpa_tl
   }
 }

\ExplSyntaxOff
\begin{document}
% node styles
\tikzstyle{flowstep} = [rectangle, draw, text centered, rounded corners, minimum height=2em]
\tikzstyle{concept}=[flowstep,fill=orange!20]
\tikzstyle{simstep}=[flowstep,fill=yellow!20]
\tikzstyle{veristep}=[flowstep,fill=BurntOrange!60]
\tikzstyle{designstep}=[flowstep,fill=pink!50]
\tikzstyle{expensivestep}=[flowstep,fill=red!30]
% edge styles
\tikzstyle{happypath}=[->,rounded corners,thick,ForestGreen!75!white]
% no arrow
\tikzstyle{unhappypathna}=[rounded corners,thick,BrickRed!75!white]
\tikzstyle{unhappypath}=[->,unhappypathna]
\tikzstyle{veryunhappypath}=[rounded corners,thick,Red,->]
\tikzset{
  fill fraction/.style n args={3}{path picture={
 \fill[#1] (path picture bounding box.south west) rectangle
 ($(path picture bounding box.north west)!#3!(path picture bounding box.north
 east)$);
 \fill[#2] (path picture bounding box.south east) rectangle
 ($(path picture bounding box.north west)!#3!(path picture bounding box.north
 east)$);}}
}

%source: andreas burg https://ee222-winter19-01.courses.soe.ucsc.edu/system/files/attachments/EE222W19_Lect1_Part2-Introduction.pdf

\begin{tikzpicture}
  % first flow nodes
  \visible<2->{
\node[flowstep](conceptidea) at (3,0) {Concept or Idea};
}
  \visible<3->{
\node[concept](projspec)[below=of conceptidea]{Project Specification};
\only<3-10>{
\draw[happypath](conceptidea) -- (projspec);
}% 
\only<11->{
  \draw[happypath](conceptidea) -- (projspec) node [midway,left]{\$};
}
}
  \visible<4->{
\node[designstep](archstep)[below=of projspec]{Architectural design}; % high level specification,break into logical blocks
}
  \visible<4->{
\node[simstep](highsimulation)[below=of projspec]{High Level Simulation}; % often adhoc
\draw[happypath](projspec) -- (highsimulation);
}
  \visible<5->{
\node[designstep](schematic)[below=of highsimulation]{Functional Design}; %Schematic
\only<5 -11>{
\draw[happypath](highsimulation) -- (schematic);
}
\only<12->{
\draw[happypath](highsimulation) -- (schematic)node [midway,left]{\$\$};
}
}
\visible<5->{
\node[simstep](simulation)[below=of schematic]{Behavioural Simulation}; % VHDL model
\draw[happypath](schematic) edge (simulation);
}
\visible<6->{
\node[flowstep,fill fraction={green!20}{blue!20}{0.5}](layout)[below=of simulation]{Layout Creation}; % can either be full custom or semi custom, will involve netlist,floor plan place and route
\only<6-11>{
\draw[happypath](simulation) -- (layout);
}
\only<12->{
\draw[happypath](simulation) -- (layout) node [midway, left,color=green!40]{\$\$} node [midway, right,color=blue!40]{{\$\$\$}};
}
}

\visible<7->{
\node[simstep](postsim)[below=of layout]{Post-Layout Simulation}; % VHDL model
\draw[happypath](layout) edge (postsim);

}
\visible<8->{
\node[expensivestep](tapeout)[below=of postsim]{Tapeout}; % VHDL model


\only<8-13>{
\draw[happypath](postsim) -- (tapeout);
}
\only<14->{
\draw[happypath](postsim) -- (tapeout) node [midway,left,color=red]{\$\$\$\$\$\$\$\$\$\$};
}

}
\visible<9->{
\node[veristep](physmeas)[below=of tapeout]{Measurements}; % VHDL model
\draw[happypath](tapeout) edge (physmeas);
}
\visible<10->{
\node[flowstep](chip) [below=of physmeas] {Functioning Chip};
\draw[happypath](physmeas) edge (chip);
}

% now happy path
%\draw[happypath](physmeas) edge (chip);
% unhappy path
\visible<13->{
% control points for  cleaner paths
\node (ctrlHS) [left=of highsimulation.west] {};
\draw[unhappypath] (highsimulation.west) -| ($(ctrlHS.east)$) |- (projspec.west);
\node (ctrlBS) [below=of ctrlHS,left=of simulation] {};
\draw[unhappypath] (simulation.west) -| (ctrlBS.center) |- (schematic.west);
\draw[unhappypath] (simulation.west) -| (ctrlBS.center) |- (projspec.west);
\node (ctrlPLS) [below=of ctrlBS,left=of postsim] {};
\draw[unhappypath] (postsim.west) -| (ctrlPLS.west) |- (layout.west);
\draw[unhappypath] (postsim.west) -| (ctrlPLS.west) |- (schematic.west);
\draw[unhappypath] (postsim.west) -| (ctrlPLS.west) |- (projspec.west);
\node (ctrlTO) [right=of tapeout] {};
\node (ctrlMeas) [below=of ctrlTO,right=of physmeas] {};
}
\visible<14->{
\draw[veryunhappypath] (physmeas.east) -|(ctrlMeas.center) |- (layout.east);
\draw[veryunhappypath] (physmeas.east) -|(ctrlMeas.center) |- (schematic.east);
\draw[veryunhappypath] (physmeas.east) -|(ctrlMeas.center) |- (projspec.east);
}
% mark project spec and functional design as topology generation as wel
\if0
\node (projtopgen) at ($(projspec.north west)+(-0.1,0.25)$) {Graph Generation!};
\draw[dashed] ($(projspec.north west)+(-0.1,0.1)$) -|($(projspec.north east)+(0.1,0.1)$)  |- ($(projspec.south east)+(0.1,-0.25)$) -| ($(projspec.south west)+(-0.10,-0.1)$) -- ($(projspec.north west)+(-0.1,0.1)$);
\fi
\if0
\node (schemtopgen) at ($(schematic.north west)+(-0.1,0.25)$) {Graph Generation!};
\draw[dashed] ($(schematic.north west)+(-0.1,0.1)$) -|($(schematic.north east)+(0.1,0.1)$)  |- ($(schematic.south east)+(0.1,-0.25)$) -| ($(schematic.south west)+(-0.10,-0.1)$) -- ($(schematic.north west)+(-0.1,0.1)$);
\fi


% *very* unhappy paths
\end{tikzpicture}
\end{document}
