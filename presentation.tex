\documentclass[11pt,xcolor={dvipsnames},hyperref={pdftex,pdfpagemode=UseNone,hidelinks,pdfdisplaydoctitle=true},usepdftitle=false]{beamer}
% load standlone first
\usepackage{presentation}
% Enter title of presentation PDF:
\hypersetup{pdftitle={PhD Defense 2024-02-28: Graph generative deep learning models with an application to circuit topologies}}
% Enter link to PDF file with figures:
\newcommand{\pdf}{figures.pdf}

\input{math.tex}
% load subfiles last
\usepackage{subfiles}
%\includeonlyframes{title,desiderata,concdesiderata}
%\includeonlyframes{title,hcgd}
%\includeonlyframes{title,digress}
%\includeonlyframes{title,workingonly}
%\includeonlyframes{title,working}
%\includeonlyframes{title,working}
%\includeonlyframes{title,working,maybedone}
%later on do
%\includeonlyframes{ttle,working,done,rest,maybedone}
\DeclarePairedDelimiter{\lpr}{(}{)}
\newcommand{\func}[2]{{#1}\lpr*{#2}}
\newcommand{\fKL}[2]{\func{\KL}{#1\Vert {#2}}}
\newcommand{\mcomm}[2][s]{\ifx s#1\ \else\ifx q\quad\fi\fi\text{\small\texttt{//#2}}}
\newcommand{\donecheck}[1]{{\large \textcolor{ForestGreen}{\checkmark}}\textcolor{black!10!white}{\emph{#1}}\xspace}
\newcommand{\missing}{{\huge\textcolor{red}{X}\xspace}}

\begin{document}
% Enter presentation title:
\title{\justifying Graph generative deep learning models with an application\\ to circuit topologies}
% Enter presentation information:
\information%
% Enter link to research paper (optional; comment line if not needed):
[Work Done: LSM@EPFL 2017-2019,LIONS@EPFL 2019-2024]%
% Enter presentation authors:
{Igor Krawczuk}%
% Enter presentation location and date (optional; comment line if not needed):
{EPF Lausanne,  2024-02-28}
\frame[label=title]{\titlepage}

\begin{frame}[label=done]
  \frametitle{Presentation Structure}
  %\framesubtitle{test}
  \centering
  \begin{enumerate}
    \item Motivation:
      \begin{enumerate}
        \item Why should care about circuit topology synthesis?
        \item Why abstracting them to graph generation is the right move?
        \item Why hasn't this been done before?
      \end{enumerate}
    %\item Background %explain why graphs weren't used before for ML in eda,explain GNNs, explain why (they were but for rule based/symbolic approaches)
    \item Contributions
      \begin{enumerate}
        \item Geometric Graph GAN
        \item DiGress
        \item Hierarchical Community Structured Graph Diffusion
      \end{enumerate}
    \item Conclusions % put the impact slide here, summarize each
  \end{enumerate} 
  \vspace{1cm} %adjust to vertically center later
\end{frame}

%introduce design flow, why do we need topology synthesis, why graphs,w hat do we need
\subfile{motivation}
%SotA at outset, start working on GG-GAN, Message passing locality assumption, power of GNNs etc.
\subfile{gg_gan}
%GG contributions: latent identifiers during training, maintain PE, collision problem, geometric identifiers is beneficial (put theorem), latency low, *FEATURES*, strong generator architecture + discriminator, *asymmetrical* geometrical embeddings!
\subfile{diffusion}
\subfile{hcgd}
%% clement reached out, we developed the model together to use the features (not possible with continuous relaxations, check, discrete state spaces allows to use GGG features=> put references, strong architecture *again*, decomposition of target distribution
%spectre put even stronger architecuter
% swingnn even stronger architctur etc.
% mahdi reached out, higen autoregressive work, developed, feature computation, ordinal diffusion, edge dependence
%maybe circuits, otherwise  just checkboxes, check asusumptions
% other related work (sparsediff)
\subfile{conclusion}

\lastslide
\subfile{backup}
\end{document}
