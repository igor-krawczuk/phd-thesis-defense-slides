% !TeX root = presentation.tex
\documentclass[./presentation.tex]{subfiles}
\begin{document}

\begin{frame}[label=concdesiderata,t]
  \frametitle{Revisiting our Requirements}
  {
  \small
    \vspace{-0.9cm}
  \begin{desideratum}
    \vspace{-0.5cm}
    \only<1->{
\begin{block}{Sample efficiency\visible<1->{\donecheck{$\SnE$ generators}}}
  Suitable inductive biases to efficiently learn generative model
\end{block}
}
    \only<2->{
  \begin{block}{Generator quality\visible<2->{\donecheck{DiGress,HCGD}}}
  Produces high fidelity samples
  \end{block}
}
    \only<3->{
  \begin{block}{Compact parametrization\visible<3->{\donecheck{GG-GAN,$\approx$HCGD}} % GG-GAN gives this, digress *kinda* gives this, hgd does again
    }
    Latent space is $\mathcal{O}\left(n\right)$ or at least subquadratic in graph size
  \end{block}
}
    \only<4->{
  \begin{block}{Latency\visible<4->{\donecheck{GG-GAN,$\approx$HCGD}} % and this, digress kinda gives this, hgd gives this
    }
  Produces sample fast enough to be used inside of optimization loops
  \end{block}
}
\only<5->{
  \begin{block}{\textbf{Scalability}\visible<5->{\donecheck{HCGD}} % ggg kinda gives this, digress kinda , hgd does
    }
  Can theoretically be extended to thousands or more nodes to approach circuit sized graphs
  \end{block}
}
\vspace{2mm}
  \end{desideratum}
}
\end{frame}
\begin{frame}[label=circdemo]
  \frametitle{Evaluation on circuits?}
  \centering
  \only<1>{
    \large Topology data remains extremely scarce\ffootnote{
    \cite{chowdhuryOpenABCDLargeScaleDataset2021e,chaiCircuitNetOpenSourceDataset2023b, panEDALearnComprehensiveRTLtoSignoff2023,weiHLSDatasetOpenSourceDataset2023}} and not readily accessible for researchers
  }
  \only<2>{
    % include the pdf, scaling it to textwidth, 0.8 height and same ratio
    \vspace{-0.5cm}
    \includegraphics[width=\textwidth,height=0.95\textheight,keepaspectratio]{./images/circuitshalleluja.pdf}
  }
  \only<3>{
    \vspace{-0.5cm}
    % include the pdf, scaling it to textwidth, 0.8 height and same ratio
    \includegraphics[width=\textwidth,height=0.95\textheight,keepaspectratio]{./images/has_learned_structure.pdf}
  }
\end{frame}
\begin{frame}[label=done]
  \frametitle{Other work}
  \vspace{-1cm}
  \begin{columns}
    \begin{column}{0.5\textwidth}
      {
  \tiny
      \begin{enumerate}
       \item[0]  \textbf{M. Karami, I. Krawczuk, V. Cevher, “Multi Resolution Graph Diffusion”, submitted to MLGenX@ICLR 2024}
       \item[1]  \textbf{C. Vignac, I. Krawczuk, A. Siraudin, B. Wang, V. Cevher, and P. Frossard, “DiGress: Discrete Denoising diffusion for graph generation.”}
      \item [2] L. Viano, A. Kamoutsi, G. Neu, I. Krawczuk, and V. Cevher, “Proximal point imitation learning,” Advances in Neural Information Processing Systems, vol. 35
      \item [3] M. Stauffer, I. Mengesha, K. Seifert, I. Krawczuk, J. Fischer, and G. Di Marzo Serugendo, “A Computational Turn in Policy Process Studies: Coevolving Network Dynamics of Policy Change,” Complexity, vol. 2022 
      \item [4] J. Sandrini, B. Attarimashalkoubeh, E. Shahrabi, I. Krawczuk, and Y. Leblebici, “Effect of metal buffer layer and thermal annealing on HfO x-based ReRAMs,” ICSEE 2016
      \item [5] T. Sanchez, I. Krawczuk, Z. Sun, and V. Cevher, “Uncertainty-Driven Adaptive Sampling via GANs,” in NeurIPS 2020 Workshop on Deep Learning and Inverse Problems, 2020 
      \item [6] T. Sanchez, I. Krawczuk, and V. Cevher, “On the benefits of deep RL in accelerated MRI sampling,” 2021, 
\end{enumerate}
}
    \end{column}
    \begin{column}{0.5\textwidth}
    {
  \tiny
      \begin{enumerate}
      \item [8] A. Ramezani-Kebrya, K. Antonakopoulos, I. Krawczuk, J. Deschenaux, and V. Cevher, “Distributed extra-gradient with optimal complexity and communication guarantees,” ICLR 2023. 
      \item [9] C. Piveteau, N. Ioannou, I. Krawczuk, M. Le Gallo-Bourdeau, A. Sebastian, and E. S. Eleftheriou, “Method for interfacing with hardware accelerators.” Patent, Feb. 15, 2022. 
      \item [10] F. Latorre, I. Krawczuk, L. T. Dadi, T. M. Pethick, and V. Cevher, “Finding actual descent directions for adversarial training,” in ICLR 2023. 
        \item [11] \textbf{I. Krawczuk, P. Abranches, A. Loukas, and V. Cevher, “GG-GAN: A geometric graph generative adversarial network, 2021 preprint 2020.} 
      \item [12] M. Brundage et al., “Toward Trustworthy AI Development: Mechanisms for Supporting Verifiable Claims.” arXiv, Apr. 20, 2020. 
      \item [13] I. Boybat et al., “Multi-ReRAM synapses for artificial neural network training,” in 2019 IEEE International Symposium on Circuits and Systems (ISCAS), IEEE, 2019, pp. 1–5. doi: 10.1109/ISCAS.2019.8702714.
      \item [14] S. Avin et al., “Filling gaps in trustworthy development of AI,” Science, vol. 374, no. 6573, pp. 1327–1329, Dec. 2021, doi: 10.1126/science.abi7176.
    \end{enumerate}
  }
    \end{column}
  \end{columns}
\end{frame}
\begin{frame}[c,label=concfin]
  \frametitle{}
  \centering
  \Huge
  \strut
 Thank you for your attention!
 \par\noindent\rule{0.8\textwidth}{0.4pt}
 Questions?
\end{frame}
\end{document}
