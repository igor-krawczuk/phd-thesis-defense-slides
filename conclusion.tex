% !TeX root = presentation.tex
\documentclass[./presentation.tex]{subfiles}
\begin{document}

\begin{frame}[label=working]
  \frametitle{Evaluation on circuits}
  \textcolor{red}{TODO: place experiments}
\end{frame}

\newcommand{\donecheck}[1]{{\huge\textcolor{ForestGreen}{\checkmark}}\emph{#1}\xspace}
\newcommand{\missing}{{\huge\textcolor{red}{X}\xspace}}
\begin{frame}[label=working,t]
  \frametitle{Why does nobody generate graphs directly yet? Revisting our {Desiderata}}
  \raggedright
  %WHY now and how will we deal with it?=> prepare for benefits of equivaraince
  Data is sparse\ffootnote{\cite{chowdhuryOpenABCDLargeScaleDataset2021e,chaiCircuitNetOpenSourceDataset2023b, panEDALearnComprehensiveRTLtoSignoff2023,weiHLSDatasetOpenSourceDataset2023}}\visible<2->{\missing{}} \emph{and} even with data, we require models that can...
\begin{itemize}
  \item model relations and compositions naturally with...\visible<2->{\donecheck{GNNs}} %GNNs/set networks give this
  \item high fidelity ...\visible<3->{\donecheck{DiGress,HCGD}} %digress, hgd gives this
  \item compact parametrization...\visible<4->{\donecheck{GG-GAN,HCGD}} % GG-GAN gives this, digress *kinda* gives this, hgd does again
  \item fast sampling...\visible<5->{\donecheck{GG-GAN,HCGD}} % and this, digress kinda gives this, hgd gives this
  \item while scaling to circuit sized graphs \visible<6->{\donecheck{HCGD}} % ggg kinda gives this, digress kinda , hgd does
\end{itemize}
\end{frame}
\begin{frame}[c]
  \frametitle{}
  \centering
  \Huge
  \strut
 Thank you for your attention!
\end{frame}
\end{document}
