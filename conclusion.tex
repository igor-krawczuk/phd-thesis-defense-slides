% !TeX root = presentation.tex
\documentclass[./presentation.tex]{subfiles}
\begin{document}

\if0
%TODO: if we manage to
\begin{frame}[label=circuitdemo]
  \frametitle{Evaluation on circuits}
  \textcolor{red}{TODO: place experiments}
\end{frame}
\fi
\begin{frame}[label=concdesiderata,t]
  \frametitle{Revisiting our Requirements}
  {
  \small
    \vspace{-0.9cm}
  \begin{desideratum}
    \vspace{-0.5cm}
    \only<1->{
\begin{block}{Sample efficiency\visible<1->{\donecheck{$\SnE$ generators}}}
  Suitable inductive biases to efficiently learn generative model
\end{block}
}
    \only<2->{
  \begin{block}{Generator quality\visible<2->{\donecheck{DiGress,HCGD}}}
  Produces high fidelity samples
  \end{block}
}
    \only<3->{
  \begin{block}{Compact parametrization\visible<3->{\donecheck{GG-GAN,$\approx$HCGD}} % GG-GAN gives this, digress *kinda* gives this, hgd does again
    }
    Latent space is $\mathcal{O}\left(n\right)$ or at least subquadratic in graph size
  \end{block}
}
    \only<4->{
  \begin{block}{Latency\visible<4->{\donecheck{GG-GAN,$\approx$HCGD}} % and this, digress kinda gives this, hgd gives this
    }
  Produces sample fast enough to be used inside of optimization loops
  \end{block}
}
\only<5->{
  \begin{block}{Scalability\visible<5->{\donecheck{HCGD}} % ggg kinda gives this, digress kinda , hgd does
    }
  Can theoretically be extended to thousands or more nodes to approach circuit sized graphs
  \end{block}
}
\vspace{2mm}
  \end{desideratum}
}
\end{frame}
\begin{frame}[c,label=concfin]
  \frametitle{}
  \centering
  \Huge
  \strut
 Thank you for your attention!
\end{frame}
\end{document}
