% !TeX root = presentation.tex
\documentclass[./presentation.tex]{subfiles}
\begin{document}
%MOTIVATION: circuit design, but we go abstract
\newcounter{myframerevealcnt}[framenumber]
%command to access value without/with an offset
\newcommand{\mfrc}{\value{myframerevealcnt}}
\newcommand{\mfrco}[1]{\numexpr{\value{myframerevealcnt}+#1}}
%step/reset myframerevealcnt
\newcommand{\stepmfrc}{\stepcounter{myframerevealcnt}}
\newcommand{\resetmfrc}{\setcounter{myframerevealcnt}{1}}
\begin{frame}[label=circuitmatter,t]
  %NOTE
  %WHY do we care about this problem
  \frametitle{Motivation: Circuit Design Matters}
  \centering
  \resetmfrc{}
  \only<\mfrc{}>{
  \vspace{-1cm}
    Worldwide Semiconductor industry revenue
  \includegraphics[height=0.7\textheight]{./images/statista_semi_market_2024-02-23_08-42.png}\\
Source: \href{}{Statista}\\
\vspace{1cm}
}
\end{frame}
\begin{frame}[label=flows1,t]
  %TODO (nice to have) slide motivating circuit design (AI chips etc
  %%TODO (nice to have) cietation for the "dark art" thing
  %%TODO (nice to have) add image of a circuit, planning etc
  %NOTE
  %WHAT does this thesis do
  \frametitle{Motivation: Circuit Design to Graph Generation}
  \centering
  \resetmfrc{}
%  \only<\mfrc{}>{Circuit Design =}
%\stepmfrc{}
\only<\mfrc{}>{
\begin{columns}
  \begin{column}{0.5\textwidth}
    \includegraphics[height=0.8\textheight]{./tikz-figs/designflow/flow.pdf}
  \end{column}
  \begin{column}{0.5\textwidth}
    \includegraphics[height=0.8\textheight]{./tikz-figs/layout-gen/layoutgen.pdf}
  \end{column}
\end{columns}
}
%%
\stepmfrc{}
\only<\mfrc{}-6>{
\only<\mfrc{}->{
  {\large 
    \uncover<1-6>{\visible<\mfrco{2}->{\emph{Automated }}\visible<\mfrc{}->{Circuit Design =\visible<0>{hackhackha}}}\\
    \only<\mfrc{}>{\only<2>{\textrightarrow}Specification +\\ Circuit Topology Design +\visible<0>{hackhackh}\\ Layout Creation}% we assume specification is a given and we implicitly condition on it
      \stepmfrc{}
      \only<\mfrc{}->{\uncover<0>{Specification +}\\}
      \only<\mfrc{}>{\only<3>{}Circuit Topology Design +\visible<0>{hackhacck}\\ Layout Creation}% human level, out of scope
      \stepmfrc{}
      \only<\mfrc{}>{Circuit Topology \emph{Generation} +\visible<0>{hackhacckckc}\\ Layout \emph{Generation}}% not technical
      \stepmfrc{}
  \only<\mfrc{}->{
  \only<\mfrc{}>{\emph{Graph} Generation}
  \only<\mfrco{1}>{$\underbrace{\text{\emph{Graph} Generation}}_{\text{this thesis}}$} +\visible<0>{hack}\\ \uncover<\mfrc{}>{
    \emph{Physics Constrained\visible<0>{ha}\\\visible<0>{ha}Graph Drawing} \ffootnote{\cite{aggarwalMultilayerGridEmbeddings1985,chungEmbeddingGraphsBooks1987,gebotysOptimalVLSIArchitectural2012,shaulyDesignRulesSemiconductor2022}}
}
}
  }%\large
}%\only
}
\only<2>{
  \\
  \includegraphics[height=0.3\textheight,width=\textwidth,keepaspectratio]{./images/spec.png}\\
  Adapted from \cite{katakkarWhatIntegratedCircuit}
}
\only<3>{
  \begin{columns}
    \begin{column}{0.5\textwidth}
      \includegraphics[height=0.5\textheight,width=\columnwidth,keepaspectratio]{./images/netlist2024-02-26_20-41.png}
      From \cite{kupriyanovHighSpeedEventDrivenRTL2004}
    \end{column}
    \begin{column}{0.5\textwidth}
      \includegraphics[height=0.5\textheight,width=\columnwidth,keepaspectratio]{./images/magiclayout.png}
      From \cite{edwardsMagicVLSI2024}
    \end{column}
  \end{columns}
}
\end{frame}

\begin{frame}[label=ruletograph,c]
  %WHAT is our approach and why is it novel
\frametitle{Topology Synthesis: From Rules to Graphs}
\vspace{-0.8cm}
% motivate why the netlist is the right abstraction level *and* why prior art didn't deal with them directly
\small
\begin{columns}
  \begin{column}{0.5\textwidth}
  {
      \begin{itemize}
        %\item industry has various subtly different variations: physical synthesis, high level synthesis,logic synthesis, topology synthesis\\\visible<2->{$\Rightarrow$ for our purposes, all the same }
        \visible<1->{
          \item Most approaches symbolic (SAT,MILP,DP) or via evolutionary strategies,strongly influenced by Church's problem \only<1->{\footnote[frame]{\small\cite{zhaoAutomatedTopologySynthesis2022a,churchApplicationRecursiveArithmetic1963} }}
        }
        %\item prior work going back to 1957 generally approaches topology generation as a special instance of Church's problem \cite{churchApplicationRecursiveArithmetic1963}, i.e. special case of program generation, solved with rule based approaches and MILP solvers or via evolutionary strategies\footnote{For a detailed overview, see e.g. \cite[section 2.2.2]{zhaoAutomatedTopologySynthesis2022a} }
          \visible<2->{
          \item Even majority of latest work\footnotemark[1] \only<2->{\ffootnote{\small\cite{hakhamaneshiBagNetBerkeleyAnalog2019d,zhaoAutomatedTopologySynthesis2022a,fayaziAnGeLFullyAutomatedAnalog2023a,dasilvaAutoTGReinforcementLearningBased2023}}} does not conceive of a circuit as a \emph{graph} but instead as a \emph{sequence of symbols} representing a hierarchy with a clear ordering
       %\item Circuit topology is modeled as a string of symbols representing circuit functionality, from which a topology is then inferred. Some work explores other representations \citep{rojecAnalogCircuitTopology2018} but majority of work \citep{fanSpecificationTopologyAutomatic2021f,zhaoAutomatedTopologySynthesis2022a,fayaziAnGeLFullyAutomatedAnalog2023a,dasilvaAutoTGReinforcementLearningBased2023} does not conceive of a circuit as a \emph{graph} but instead as a \emph{sequence of hierarchical symbols} with a clear ordering
        }
          \visible<3->{
          \item Single Exception: \cite{fanSpecificationTopologyAutomatic2021d}
          }
     \end{itemize}
   }
  \end{column}%
  \begin{column}{0.5\textwidth}
  \centering
  \only<1>{
  \includegraphics[width=\columnwidth,height=0.80\textheight,keepaspectratio]{./images/oasys_2024-02-24_21-59.png}%netlist code,textual vis
  From: \cite{harjaniOASYSFrameworkAnalog1989a}
}
  \only<2>{
  \includegraphics[width=\columnwidth,height=0.80\textheight,keepaspectratio]{./images/graph_grammar_2024-02-24_21-39.png}
  From: \cite{zhaoGraphGrammarBasedAnalogCircuit2019}
}
  \only<3>{
  \includegraphics[width=\columnwidth,height=0.70\textheight,keepaspectratio]{./images/single_graph_related_2024-02-24_22-09.png}
  From: \cite{fanSpecificationTopologyAutomatic2021d}
}
  \end{column}
\end{columns}
\end{frame}
\begin{frame}[label=ruletograph,t]
  \frametitle{Why no learned graph generation for topology synthesis?}
  \centering
  %WHY now and how will we deal with it?=> prepare for benefits of equivaraince
  \only<1>{
    \includegraphics[width=\textwidth,height=0.6\textheight,keepaspectratio]{./images/eda_tools_2024-02-24_22-21.png}
    \cite{loperaSurveyGraphNeural2021d}}
\end{frame}
\begin{frame}[label=flows2,t]
  %Why use graphs? does this thesis do
  \frametitle{Circuit Design to Graph Generation}
  \begin{columns}
  \begin{column}{0.5\textwidth}
    \only<1>{\includegraphics[height=0.8\textheight]{./tikz-figs/designflow/flow.pdf}}
    % these last 2 *briefly*, don't overfocus, since we need to rush through them
    %\only<8>{\includegraphics[height=0.8\textheight]{./tikz-figs/designflow/flow_with_proj.pdf}}
      \only<2>{\includegraphics[height=0.8\textheight]{./tikz-figs/designflow/flow_with_both.pdf}}
  \end{column}
  \begin{column}{0.5\textwidth}
    %\only<5>{\includegraphics[height=0.8\textheight]{./tikz-figs/layout-gen/layoutgen.pdf}}
    %\only<6>{\includegraphics[height=0.8\textheight]{./tikz-figs/layout-gen/layoutgen_with_topsyn.pdf}}
    \only<1->{\includegraphics[height=0.8\textheight]{./tikz-figs/layout-gen/layoutgen_with_graphgen.pdf}}
  \end{column}
  \end{columns}
%TODO: maybe add images as interludes in the automation for constraints
\end{frame}
\begin{frame}[label=predesiderata,t]
  \frametitle{Why no learned graph generation for topology synthesis?}
  \begin{itemize}
    \item Simple RL isn't tractable due to combinatorial complexity 
      \only<2->{
      \item To improve this we need data for pretraining\ffootnote{\cite{gulcehreKnowledgeMattersImportance2016,paineMakingEfficientUse2019a,schwarzerPretrainingRepresentationsDataEfficient2021,agarwalReincarnatingReinforcementLearning2022a}} and parametrizing agents with generative models\ffootnote{\cite{vlassisBayesianReinforcementLearning2012a,buesingLearningQueryingFast2018,sarmadRLGANNetReinforcementLearning2019b}}
      }
      \only<3->{
    \item Topology data is extremely scarce\ffootnote{\cite{chowdhuryOpenABCDLargeScaleDataset2021e,chaiCircuitNetOpenSourceDataset2023b, panEDALearnComprehensiveRTLtoSignoff2023,weiHLSDatasetOpenSourceDataset2023}} and not readily accessible for researchers
    }
    \only<4->{
    \item Even with data, we require suitable generative models to parametrize!
    }
  \end{itemize}
\end{frame}

\begin{frame}[label=desiderata,t]
  \frametitle{Requirements to apply Deep graph generators to circuit topologies}
  {
  \small
    \vspace{-0.9cm}
  \begin{desideratum}
    \vspace{-0.5cm}
  \only<1->{
\begin{block}{Sample efficiency}
  Suitable inductive biases to efficiently learn generative model
\end{block}
}
  \only<2->{
  \begin{block}{Generator quality}
  Produces high fidelity samples
  \end{block}
}
  \only<3->{
  \begin{block}{Compact parametrization}
    Latent space is $\mathcal{O}\left(n\right)$ or at least subquadratic in graph size
  \end{block}
}
  \only<4->{
  \begin{block}{Latency}
  Produces sample fast enough to be used inside of optimization loops
  \end{block}
}
  \only<5->{
  \begin{block}{Scalability}
  Can theoretically be extended to thousands or more nodes to approach circuit sized graphs
  \end{block}
}
\vspace{2mm}
  \end{desideratum}

}
\end{frame}

\end{document}

