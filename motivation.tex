% !TeX root = presentation.tex
\documentclass[./presentation.tex]{subfiles}
\begin{document}

\begin{frame}[label=maybedone]
\frametitle{EDA Flow Overview}
  \begin{columns}
    \begin{column}{0.5\textwidth}
      \centering
      \resizebox{!}{0.8\textheight}{
      \subfile{./tikz-figs/designflow/flow.tex}
    }
    \end{column}
    \begin{column}{0.5\textwidth}
      \centering
    \resizebox{\columnwidth}{!}{%0.8\textheight}{
      \subfile{./tikz-figs/layout-gen/layoutgen.tex}
    }
    \end{column}
  \end{columns}
\end{frame}

\begin{frame}[label=todo]
\frametitle{Topology Synthesis: From Macros to Graphs}
% motivate why the netlist is the right abstraction level *and* why prior art didn't deal with them directly
\begin{columns}
  {\small
  \begin{column}{0.5\textwidth}
   \begin{itemize}
     \item industry has various subtly different variations: physical synthesis, high level synthesis,logic synthesis, topology synthesis\\\visible<2->{$\Rightarrow$ for our purposes, all the same }
     \item prior work going back to 1957 generally approaches synthesis as a special instance of Church's problem \cite{churchApplicationRecursiveArithmetic1963}, i.e. special case of program synthesis.
     \item Circuit topology is modeled as a string of symbols representing circuit functionality, from which a topology is then inferred. Some work explores other representations \cite{rojecAnalogCircuitTopology2018} but fundamentally even latest work \cite{fayaziAnGeLFullyAutomatedAnalog2023a,dasilvaAutoTGReinforcementLearningBased2023,fanSpecificationTopologyAutomatic2021f,zhaoAutomatedTopologySynthesis2022a} work does not conceive of a circuit as a single \emph{graph} but instead as a \emph{sequence of hierarchical symbols} with a clear ordering
   \end{itemize} 
  \end{column}
}
  \begin{column}{0.5\textwidth}
  \centering
  \includegraphics[width=\columnwidth,height=0.45\textheight]{example-image-a}\\%netlist code,textual vis}
  \includegraphics[width=\columnwidth,height=0.45\textheight]{example-image-b}%netlist graph}
  \end{column}
\end{columns}
\end{frame}
\begin{frame}[label=todo]
\frametitle{Why Represent Topologies as Graphs}
\begin{itemize}
  \item automatically encode topological isomorphisms (note: might not be enough for \emph{circuit} isomorphisms \cite{wangFunctionalityMattersNetlist2022c}
  \item naturally captures relational and compositional structure
  \item amenable to preserving notions of locality non-planar, high dimensional structure (important for multi-layer layouting)
  \item naturally abstracts the level in the hiearchy away and subsumes prior modeling - building blocks, devices, subcircuits all become nodes
\end{itemize}
\end{frame}

\begin{frame}[label=working]
  \frametitle{Netlists are Graphs are Graphs}
  \begin{itemize}
    \item For the purpose of the thesis, a circuit $\mathcal{C}$ is just a graph $\mathcal{G}\coloneq \lbrace\mathcal{E},\mathval{V},\mathcal{Y} \rbrace$.
    \item In general,we represent each edge 
  \end{itemize}
\end{frame}

\begin{frame}[label=working]
\frametitle{SotA Topology Synthesis}
\begin{itemize}
  \item show \cite{zhaoAutomatedTopologySynthesis2022a} work, comment on it in images
  \item show BAG, other sota
\end{itemize}
\end{frame}

\begin{frame}[label=working]
\frametitle{Desiderata}
\begin{itemize}
  \item foo
\end{itemize}
\end{frame}

  
\end{document}

