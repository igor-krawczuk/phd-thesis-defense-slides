% !TeX root = presentation.tex
\documentclass[./presentation.tex]{subfiles}
\begin{document}
%MOTIVATION: circuit design, but we go abstract
\begin{frame}[label=circuitmatter,t]
  %NOTE
  %WHY do we care about this problem
  \frametitle{Motivation: Circuit Design Matters}
  \centering
  \resetmfrc{}
  \only<\mfrc{}>{
  \vspace{-1cm}
    Worldwide Semiconductor industry revenue
  \includegraphics[height=0.7\textheight]{./images/statista_semi_market_2024-02-23_08-42.png}\\
Source: \href{}{Statista}\\
\vspace{1cm}
}
\end{frame}
\begin{frame}[label=flowsabstraction,t]
  \frametitle{Motivation: Circuit Design to Graph Generation}
  \centering
  \only<3-4>{
  \vspace{-0.8cm}
}
\only<1->{
  {\large 
    \uncover<1-5>{\visible<5->{\emph{Automated }}\visible<1->{Circuit Design =\visible<0>{hackhackha}}}\\
    \only<2-3>{\only<3>{\textrightarrow}Specification +\\ Circuit Topology Design +\visible<0>{hackhackh}\\ Layout Creation}% we assume specification is a given and we implicitly condition on it
      \only<4->{\uncover<0>{Specification +}\\}
      \only<3-4>{\only<3-4>{}Circuit Topology Design +\visible<0>{hackhacck}\\ Layout Creation}% human level, out of scope
      \only<5>{Circuit Topology \emph{Generation} +\visible<0>{hackhacckckc}\\ Layout \emph{Generation}}% not technical
  \only<6->{
  \only<6>{\emph{Graph} Generation}
  \only<7>{$\underbrace{\text{\emph{Graph} Generation}}_{\text{this thesis}}$} +\visible<0>{hack}\\ \uncover<6>{
    \emph{Physics Constrained\visible<0>{ha}\\\visible<0>{ha}Graph Drawing} \ffootnote{\cite{aggarwalMultilayerGridEmbeddings1985,chungEmbeddingGraphsBooks1987,gebotysOptimalVLSIArchitectural2012,shaulyDesignRulesSemiconductor2022}}
}
  }%\large
}%\only
}
\only<3>{
  \\
  \includegraphics[height=0.3\textheight,width=\textwidth,keepaspectratio]{./images/spec.png}\\
  Adapted from \cite{katakkarWhatIntegratedCircuit}
}
\only<4>{
  \begin{columns}
    \begin{column}{0.5\textwidth}
      \includegraphics[height=0.5\textheight,width=\columnwidth,keepaspectratio]{./images/netlist2024-02-26_20-41.png}
      From \cite{kupriyanovHighSpeedEventDrivenRTL2004}
    \end{column}
    \begin{column}{0.5\textwidth}
      \includegraphics[height=0.5\textheight,width=\columnwidth,keepaspectratio]{./images/magiclayout.png}\\
      From \cite{edwardsMagicVLSI2024}
    \end{column}
  \end{columns}
}
\end{frame}


\begin{frame}[label=ruletograph,c]
  %WHAT is our approach and why is it novel
\frametitle{Topology Synthesis: All rewriting, no graphs?}
\vspace{-0.8cm}
% motivate why the netlist is the right abstraction level *and* why prior art didn't deal with them directly
\small
\begin{columns}
  \begin{column}{0.5\textwidth}
     \begin{itemize}
          \item Most approaches symbolic (SAT,MILP,DP) or via evolutionary strategies,strongly influenced by Church's problem \only<1->{\footnote[frame]{\small\cite{zhaoAutomatedTopologySynthesis2022a,churchApplicationRecursiveArithmetic1963} }},including majority of latest work\footnotemark[1] \only<1->{\ffootnote{\small\cite{hakhamaneshiBagNetBerkeleyAnalog2019d,zhaoAutomatedTopologySynthesis2022a,fayaziAnGeLFullyAutomatedAnalog2023a,dasilvaAutoTGReinforcementLearningBased2023}}} %does not conceive of a circuit as a \emph{graph} but instead as a \emph{sequence of symbols} representing a hierarchy with a clear ordering
        
          \item Single Exception: \cite{fanSpecificationTopologyAutomatic2021d}
     \end{itemize}
  \end{column}%
  \begin{column}{0.5\textwidth}
  \centering
  \includegraphics[width=\columnwidth,height=0.70\textheight,keepaspectratio]{./images/single_graph_related_2024-02-24_22-09.png}\\
  From: \cite{fanSpecificationTopologyAutomatic2021d}
  \end{column}
\end{columns}
\end{frame}
\begin{frame}[label=predesiderata,t]
  \frametitle{Why no learned graph generation for topology synthesis?}
  \begin{itemize}
    \item Simple RL isn't tractable due to combinatorial complexity \ffootnote{\cite{fanSpecificationTopologyAutomatic2021d}}
      \only<2->{
      \item To improve this we need data for pretraining\ffootnote{\cite{gulcehreKnowledgeMattersImportance2016,paineMakingEfficientUse2019a,schwarzerPretrainingRepresentationsDataEfficient2021,agarwalReincarnatingReinforcementLearning2022a}} and parametrizing agents with generative models\ffootnote{\cite{vlassisBayesianReinforcementLearning2012a,buesingLearningQueryingFast2018,sarmadRLGANNetReinforcementLearning2019b}}
      }
      \only<3->{
    \item Topology data is extremely scarce\ffootnote{\cite{chowdhuryOpenABCDLargeScaleDataset2021e,chaiCircuitNetOpenSourceDataset2023b, panEDALearnComprehensiveRTLtoSignoff2023,weiHLSDatasetOpenSourceDataset2023}} and not readily accessible for researchers
    }
    \only<4->{
    \item \only<4>{Even with data, we require suitable generative models to parametrize!}
    \only<5>{
      \textbf{Even with data, we require suitable generative models to parametrize!}$\Leftarrow$\emph{this thesis}
  }
  }
  \end{itemize}
\end{frame}

\begin{frame}[label=desiderata,t]
  \frametitle{Requirements to apply Deep graph generators to circuit topologies}
  {
  \small
    \vspace{-0.9cm}
  \begin{desideratum}
    \vspace{-0.5cm}
  \only<1->{
\begin{block}{Sample efficiency}
  Suitable inductive biases to efficiently learn generative model
\end{block}
}
  \only<2->{
  \begin{block}{Generator quality}
  Produces high fidelity samples
  \end{block}
}
  \only<3->{
  \begin{block}{Compact parametrization}
    Latent space is $\mathcal{O}\left(n\right)$ or at least subquadratic in graph size
  \end{block}
}
  \only<4->{
  \begin{block}{Latency}
  Produces sample fast enough to be used inside of optimization loops
  \end{block}
}
  \only<5->{
    \begin{block}{\textbf{Scalability}}
  Can theoretically be extended to thousands or more nodes to approach circuit sized graphs
  \end{block}
}
\vspace{2mm}
  \end{desideratum}

}
\end{frame}

\end{document}

