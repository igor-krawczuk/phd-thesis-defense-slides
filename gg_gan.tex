% !TeX root = presentation.tex
\documentclass[./presentation.tex]{subfiles}
\begin{document}
\begin{frame}[label=working,t]
  \frametitle{Why does nobody generate graphs directly yet? \visible<3->{Desiderata}
  }
  \centering
  %WHY now and how will we deal with it?=> prepare for benefits of equivaraince
\only<1-2>{
  GNNs are very effective for EDA tasks \ffootnote{\cite{renWhyAreGraph2022b}}\\
  \only<1>{
    \includegraphics[width=\textwidth,height=0.6\textheight,keepaspectratio]{./images/effective2024-02-24_22-24.png}
  }
  \only<2>{
    So why no topology generation yet\ffootnote{\cite{loperaSurveyGraphNeural2021d}}?\\
    \includegraphics[width=\textwidth,height=0.6\textheight,keepaspectratio]{./images/eda_tools_2024-02-24_22-21.png}
  }
}
\only<3->{
  \raggedright
  Require models that can...
\begin{itemize}
    \visible<4->{\item model relations and compositions naturally with...} %GNNs/set networks give this
    \visible<5->{\item high fidelity ...} %digress, hgd gives this
    \visible<6->{\item compact parametrization...} % GG-GAN gives this, digress *kinda* gives this, hgd does again
    \visible<7->{\item fast sampling...} % and this, digress kinda gives this, hgd gives this
    \visible<8->{\item while scaling to circuit sized graphs } % ggg kinda gives this, digress kinda , hgd does
\end{itemize}
}
\end{frame}
\begin{frame}[label=working,t]
%% WP1
  \frametitle{Machine Learning for Relations: GNNs \& Geometric DL}
  \only<1>{
    \includegraphics[width=\textwidth,height=0.8\textheight,keepaspectratio]{./images/rise_of_gnns_2024-02-24_23-01.png}\\
  }
  \only<2-6>{
        \begin{itemize}
          \item Graph Neural Networks: learn features representing structure of graphs\ffootnote{\cite{scarselliGraphNeuralNetwork2009}}\ffootnote{\cite{duvenaudConvolutionalNetworksGraphs2015,kipfSemiSupervisedClassificationGraph2017b,gilmerNeuralMessagePassing2017a,battagliaRelationalInductiveBiases2018k}}
          \only<3-4>{
        \item general MPNN form\ffootnote{\cite{xuHowPowerfulAre2019e}}}
          \only<5->{
          \item important for modern GNNs\footnotemark[4]: permutation equivariant/invariant parametrization of learnable functions on graphs!
          }
          \only<6->{
          \item $\mathcal{O}\left(n\right)$ but expressivity constraints\footnotemark[5], $\mathcal{O}\left(n^3\right)$ for more expressive models \ffootnote{\cite{maronProvablyPowerfulGraph2019b,vignacBuildingPowerfulEquivariant2020,balcilarBreakingLimitsMessage2021b}}% either in model for for pre-computation
          }
        \end{itemize}
        \only<3>{
          \begin{align}
            h_{x_i}^{(l+1)}=&\combine^{(l)}\left(h_{x_i}^{(l)},\aggregate^{(l)}\left(\lbrace h_{x_{j}}^{(l)} \forall x_j \in \mathcal{N}\left(x_i\right)\rbrace\right)\right)\\
          z_\mathcal{G}=&\readout\left(\lbrace h_{v_i}^{(l)} \forall v_i \in \mathcal{G} \rbrace\right)
        \end{align}
      }
        \only<4>{
          \begin{align}
            h_{v_i}^{0}=&x_i&\text{\texttt{//initial node features}}\nonumber\\
            h_{v_i}^{(l+1)}=&MLP^{(l)}\left(\left(1+\epsilon^{(l)}\right)h_{v_i}^{(l)}+\sum_{x_j \in \mathcal{N}\left(x_i\right)}h_{v_{j}^{(l)}}\right)&\text{\texttt{//per layer update}}\nonumber\\
            z_\mathcal{G}=&\concat_{\forall l}\left(\sum_{v_i\in \mathcal{G}}h_{v_i}^{(l)} \right)&\text{\texttt{//final readout}}\nonumber
        \end{align}
      }
  }
  \only<7->{
    \textbf{Invariance,Equivariance,Symmetry}: A function $f:\mathcal{X}\to\mathcal{Y}$ is $G$-\emph{invariant} if 

    \begin{equation}
  f(\phi(g)x)=f(x)\forall x \in \mathcal{X},g\in G\nonumber
    \end{equation}

  and G-\emph{equivariant} if
  \begin{equation}
  f(\phi(g)x)=\rho(g)f(x) \forall x \in\mathcal{X},g\in G\nonumber
  \end{equation}

  If $f(x;\theta)$ is G-(in-/equi)-variant, $\nabla_\theta f(\phi(g)x;\theta)=\nabla_\theta f(x;\theta)$\\

  $\implies$ the network does not need to learn variation capture by $G$ improving sample complexity for generalizeable learning
  \footnote[frame]{\cite{elesedyProvablyStrictGeneralisation2021b}}, although caveats apply\ffootnote{\cite{abbeNonuniversalityDeepLearning2022c,kianiHardnessLearningSymmetries2024b}}
}
\end{frame}

\begin{frame}[label=working]
  \frametitle{SotA Deep Graph Generative Models at outset (2019)}
  \small
  4 main variants
  \begin{itemize}
    \item Autoregressive likelihood(AR)\ffootnote{ GraphRNN, GRAN, BiGG} $\centernot\implies$ order dependent,high latency%TODO cite the paper that shows that order matters
    \item GAN \ffootnote{\cite{decaoMolGANImplicitGenerative2022b}}% GAN, non PE generator, fixed size matrix  
      $\centernot\implies$ non-PE generator, no size extrapolation,mode collapse
    \item VAE/VAE-GAN hybrid\ffootnote{\cite{yangConditionalStructureGeneration2019e,kipfVariationalGraphAutoEncoders2016b}}% V FNN generator
      $\centernot\implies$ non-PE generator, no size extrapolation,VAE limitations \ffootnote{\cite{bousquetOptimalTransportGenerative2017a, genevayGANVAEOptimal2017e}}%we want to be blind to permutation *once done*; during *training* we can impose orderings/identities as long as it happens in the latent state and agnostic to the data?
    \item Normalizing Flows\ffootnote{\cite{liuGraphNormalizingFlows2019a,madhawaGraphNVPInvertibleFlow2019a}}$\centernot\implies$ PE, slow training, limited expressivity
    \item Score-Matching + Langevin Dynamics\ffootnote{\cite{niuPermutationInvariantGraph2020b}}  $\centernot\implies$ PE,but very slow sampling
  \end{itemize}
\end{frame}

\begin{frame}[label=working,t]
  \frametitle{Geometric Graph Generation}
  %need to explain this
  %https://en.wikipedia.org/wiki/Spatial_network
  %threshold graph
  
\end{frame}

%PROBLEM: no big graphs=> why?

\begin{frame}[label=working,t]
  \frametitle{Collision problem}
  
\end{frame}
\begin{frame}[label=working,t]
  \frametitle{Resolution: Latent identifiers}
  
\end{frame}
\begin{frame}[label=working,t]
  \frametitle{Results}
  
\end{frame}
\begin{frame}[label=working,t]
  \frametitle{Results (latency)}
  
\end{frame}

%backup slide: asymmetrical embeddings are important

\begin{frame}[label=working,t]
  \frametitle{Impact}
  \begin{itemize}
    \item strengths: low latency\checkmark, compact\checkmark,scaling to medium graphs\checkmark, weaknesses: training stability,expressiveness, scaling to \emph{very}large graphs
    \item \citep{vignacTopNEquivariantSet2021d} further analyzed the empirical observations, the impact of $\Phi$ and proposed an alternative latent parametrization
    \item \citep{martinkusSPECTRESpectralConditioning2022b} further improve the model with more powerful models and spectral conditioning, setting a new SotA in structure generation
  \end{itemize}
\end{frame}

\end{document}

