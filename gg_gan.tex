% !TeX root = presentation.tex
\documentclass[./presentation.tex]{subfiles}
\begin{document}
\begin{frame}[label=working]
  \frametitle{SotA Graph Generative Models at outset (2019)}
  \begin{itemize}
    \item GraphRNN, GRAN, BiGG: autoregressive likelihood $\implies$ order dependent,high latency
    \item MolGAN: GAN, non PE generator, fixed size matrix  $\implies$ no size extrapolation,mode collapse
    \item CondGEN: VAE-GAN hybrid, FNN generator $\implies$ VAE limitations \citep{bousquetOptimalTransportGenerative2017a, genevayGANVAEOptimal2017e}
  \end{itemize}
\end{frame}

\begin{frame}[label=working]
  \frametitle{Equivariance \& Geometric DL}
\end{frame}
%% WP1
\begin{frame}[label=working,t]
  \frametitle{Geometric Graph Generation}
  
\end{frame}

\begin{frame}[label=working,t]
  \frametitle{Collision problem}
  
\end{frame}
\begin{frame}[label=working,t]
  \frametitle{Resolution: Latent identifiers}
  
\end{frame}
\begin{frame}[label=working,t]
  \frametitle{Results}
  
\end{frame}
\begin{frame}[label=working,t]
  \frametitle{Results (latency)}
  
\end{frame}

%backup slide: asymmetrical embeddings are important

\begin{frame}[label=working,t]
  \frametitle{Impact}
  \begin{itemize}
    \item strengths: low latency\checkmark, compact\checkmark,scaling to medium graphs\checkmark, weaknesses: training stability,expressiveness, scaling to \emph{very}large graphs
    \item \citep{vignacTopNEquivariantSet2021d} further analyzed the empirical observations, the impact of $\Phi$ and proposed an alternative latent parametrization
    \item \citep{martinkusSPECTRESpectralConditioning2022b} further improve the model with more powerful models and spectral conditioning, setting a new SotA in structure generation
  \end{itemize}
\end{frame}

\end{document}

